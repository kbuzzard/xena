\documentclass[10pt]{article}
\usepackage{amsfonts}
\usepackage{a4wide}
\usepackage{amsmath}
\thispagestyle{empty}
% for fancy 2 column lists with letters
\usepackage{multicol}
\usepackage[shortlabels]{enumitem}
\newcommand{\R}{\mathbf{R}}
\newcommand{\Z}{\mathbf{Z}}
\begin{document}


\begin{flushright} KMB,\ 31/10/17\end{flushright}

\noindent{\large \bf M1F Foundations of Analysis, Problem Sheet 4.}

\medskip
\noindent{\bf 1.} I will remind you, before we start on this question, that you're supposed to be discussing the starred parts of the sheet with your personal tutor each week.

a) Let $p=a+ib$ be a complex number. Draw a picture to show that $\overline{p}$ equals the reflection of $p$ in the real axis.

b) By thinking of complex conjugation as reflection, and by thinking of addition of complex numbers as addition of vectors, draw a little picture to convince yourself that for $p$ and $q$ complex numbers, the statement
$$\overline{p+q}=\overline{p}+\overline{q}$$
is obvious.

c) By using the picture you drew in part (a), convince yourself that $\overline{re^{i\theta}}=re^{-i\theta}$. 

d) We know from de Moivre that to multiply by $re^{i\theta}$ we first scale by $r$ and then rotate by $\theta$. Convince yourself, thinking geometrically, that
$$\overline{pq}=\overline{p}\,\overline{q}$$
is obvious.

e${}^*$) Do you think that your answers to (b) and (d) are rigorous mathematical proofs? Do you think that your answers are even mathematics? If not, what do you think they are?

\medskip
\noindent{\bf 2.} Recall that we showed in lectures that $\cos(3\theta)=4c^3-3c$, where $c=\cos(\theta)$. Here's how we can use this fact to solve a cubic equation! Rather than getting bogged down with $Ax^3+Bx^2+Cx+D=0$, let me just use numbers; the technique will work in general (kind of\ldots).

a) Pull out your calculator -- we're going to find the roots of $3x^3 - 18x^2 + 27x - 4=0$ using it. Make sure the cosine button is working (and the inverse cosine button).

b) First we do a linear change of variables to kill the $x^2$ term. So set $y=x-2$ (the point being that then $3y^3=3x^3 - 18x^2 + \cdots$) and rewrite the cubic equation as a cubic equation in~$y$ instead (don't forget the $=0$ bit, that's an important part of the equation).

c) Now we want to scale $y$ to make that cubic equation look like $4c^3-3c+\cdots=0$, and a bit of playing around should convince you that one way of doing this is by setting $c=y/2$ and then dividing the entire equation by 6.

Spoiler: if you've got it all right so far, you should have
$$4c^3 - 3c + 1/3=0.$$

d) Now substitute $c=\cos(\theta)$ and deduce that we want to solve $\cos(3\theta)=-1/3$. Solve this for $\theta$ using your calculator and hence work out $c$, $y$ and then $x$.

e) Did it work? It did for me, I got $x=3.6079128829148322904316053206617018144\ldots$ which does seem to be a root.

f) Aren't cubics supposed to have three roots? Can you get all three using this method?

g) Can you solve all quartics this way?

\medskip\noindent{\bf 3.} What is $\binom{100}{0}-\binom{100}{2}+\binom{100}{4}-\binom{100}{6}+\cdots+\binom{100}{100}$? Hint: do you remember me running into
a question like this in lectures? [Whenever I was doing problem sheets like this, I would sometimes just mindlessly page through the notes I'd been taking for the course, looking for inspiration\ldots]

\medskip\noindent{\bf 4.} 

(a) By considering $(1+i)(\sqrt{3}-i)$ or otherwise, prove that $\cos(\pi/12)=\frac{\sqrt{6}+\sqrt{2}}{4}$. 

(b) Deduce that $\cos(\pi/12)$ is irrational. 

NB for those of you who never got the hang of radians, $\pi/12$ is 15 degrees and I'm trying to get you to use the amazing insight that $45-30=15$. But honestly, don't be like me, get the hang of radians.

\medskip\noindent{\bf 5.} (a) Draw a picture of the ten 10th roots of $i$ in the complex plane. Which one is closest to $i$?

(b) Let~$z$ be a non-zero complex number. Prove that the three cube roots of~$z$ in the complex plane are at the vertices of an equilateral triangle.
\end{document}
